\chapter*{Abstract}\label{chapter:abstract}

Somewhat ominously, we are getting closer and closer to ubiquitous remote sensing and gesture recognition through the use of only Wi-Fi signals.
This is understood as the ability to recognize gestures performed by an individual with nothing more than the interference their bodies provide to surrounding Wi-Fi signals.
The main challenge facing the mainstream adoption of such technologies is the lack of generalizability seen in published models against various domain factors.
This work aims to bring us closer to such a (dystopian) future where such technologies may be mainstream by presenting DARLInG, a novel approach to domain shift mitigation through the use of domain auto-labeling using reinforcement learning.
We show that this may be a promising approach for domain shift mitigation, although further research is necessary to investigate whether this approach will actually be applicable in mainstream applications.