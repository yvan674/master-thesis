\chapter*{Abstract}\label{chapter:abstract}

Somewhat ominously, we are getting closer and closer to ubiquitous remote sensing and gesture recognition through the use of only Wi-Fi signals.
This is understood as the ability to recognize gestures performed by an individual with nothing more than the interference their bodies cause to surrounding Wi-Fi signals.
The main challenge facing the mainstream adoption of such technologies is the lack of generalizability seen in published models against various domain factors.
This work aims to bring us closer to such a (dystopian) future where such technologies may be mainstream by presenting DARLInG (Domain Autolabeling through Reinforcement Learning for the Inference of Gestures), a novel approach to domain shift mitigation through the use of domain auto-labeling using reinforcement learning.
In this work, we explore DARLInG and show that while promising, further research will be necessary to investigate the true extent of the capabilities of our approach.