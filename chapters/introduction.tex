\chapter{Introduction}\label{chapter:introduction}


\begin{itemize}
	\item Developments in IoT applications
	\item Developments in ubiquitous computing
	\item Developments in ubiquitous sensing and embedded sensors
	\item Challenges in a user interface for ubiquitous computing
	\item Wi-Fi sensing as a potential solution
\end{itemize}

With this thesis, we aim to explore the use of CNN architectures and domain-shift mitigation methods to improve the state-of-the-art in Wi-Fi CSI-based gesture classification.

\section{Context and Background}\label{sec:intro-context}

\begin{itemize}
	\item IoT devices and ubiquitous computing has become quite common and are convenient
	\item Main issue is how to provide some sort of always-available interface
	\item Sci-Fi presents gesture-based interfaces, but this requires ubiquitous sensors
	\item Issue with learning based approaches is that it often suffers from an inability to adapt to domain shifts
	\item In ubiquitous sensing, minimal setup is required on the user's part, otherwise it won't become mainstream if it is tedious to set up
	\item Why is Wi-Fi a potential solution to a low-cost ubiquitous sensing device
	\item 802.11bf, sensing standardization using Wi-Fi shows this might become common in the future and is being taken seriously by multiple large industry stakeholders
\end{itemize}

Wi-Fi technology, when boiled down, is just a really complex radio and what is radar but a different form of very complex radio.
It naturally, or not so naturally, follows then, can Wi-Fi be used for remote sensing analogously to radar technology?
The answer to this question, according to \cite{adib2013see} and \cite{chetty2011through}, the answer is a resounding yes!

\section{Motivation}\label{sec:intro-motivation}

\begin{itemize}
	\item Potential IoT/smart home applications
	\item To provide a ubiquitous sensing system which is low-cost and already prevalent in many environments
	\item To investigate and advance domain-agnostic learning systems
	\item To further the state-of-the-art for gesture classification
	\item If 802.11bf comes to fruition, and we want to take advantage of it, we should start now
\end{itemize}

\section{Problem Statement}\label{sec:intro-problem-statement}

\begin{itemize}
	\item The potential for Wi-Fi to be a modality for ubiquitous sensing should not be underestimated.
	\item Its prevalence in almost every modern building and home shows how widespread the technology already is.
	\item Domain-agnostic models exist, but are not as good as specialiized models
	\item The issue mostly boils down to the lack of generalizability of these existing models and the lack of ability to deal with domain-shift in any learning-based application
	\item What solutions may exist to improve performance
\end{itemize}

We aim to improve the state-of-the-art results in domain-agnostic gesture classification.
We first clean the raw CSI signals using techniques common in radar technology and from the literature, providing a cleaner signal for the model to work with.
We then use table-to-image transformations to allow for our model to use images as its input, utilizing advances in learning-based image-processing algorithms.
Finally, we utilize a reinforcement learning domain-recognition approach to provide our classification model with a latent representation of the domain, providing it with additional information to improve its prediction performance.


